This chapter provides one quick way of installing \Dumux.
You should have a recent working Linux environment and no \Dune core modules should be installed.
If you need more information or
have \Dune already installed, please have a look at the detailed installation
instructions in the more detailed intructions in the next chapter \ref{detailed-install}.

\section{Prerequisites} \label{sec:prerequisites}
For this quick start guide the following software packages are required:
\begin{itemize}
\item GitLab client
\item A standard compliant C++ compiler supporting C++11 and the C++14 feature set of GCC 4.9. We support GCC 4.9 or newer and Clang 3.8 or newer.
\item CMake 2.8.12 or newer
\item pkg-config
\item paraview (to visualize the results)
\end{itemize}

\section{Obtaining code and configuring all modules with a script}
We provide you with a shell-script \texttt{installDumux.sh} that facilitates setting up a {\Dune}/{\Dumux} directory tree
and configures all modules with CMake.
% TODO: uncomment/delete the following lines when next is the only release
% It is available after obtaining a download link via \url{http://www.dumux.org/download/} or
% by copying the following lines into a text-file named \texttt{installDumux.sh}:
Copy the following lines into a text-file named \texttt{installDumux.sh}:
\lstinputlisting[style=DumuxCode, numbersep=5pt, firstline=1, firstnumber=1]{installDumux.sh}

Place the \texttt{installDumux.sh} script in the directory where you want to install \Dumux and \Dune (a single
root folder \texttt{DUMUX} will be produced, so you do not need to provide one).
Run the script by typing into the terminal: \texttt{./installDumux.sh}

Configuring \Dune and \Dumux is done by the command-line script \texttt{dunecontrol}
using optimized configure options, see the line entitled \texttt{\# run build} in the \texttt{installDumux.sh} script.
More details about the build-system can be found in section \ref{buildIt}.

\subsection{A first test run of \Dumux}
When the \texttt{installDumux.sh} script from the subsection above has run successfully, you can execute a second script that
will compile and run a simple one-phase ground water flow example and will visualize the result using ParaView.
The test script can be obtained by copying the following lines into a text-file named \texttt{test\_dumux.sh}
that has to be located in the same directory as the installation script.
\begin{lstlisting}[style=DumuxCode]
cd DUMUX/dumux/build-cmake/test/porousmediumflow/1p/implicit
make -B test_1pcctpfa
./test_1pcctpfa test_1pcctpfa.input
paraview *pvd
\end{lstlisting}
The script \texttt{test\_dumux.sh} can be executed by typing into the terminal: \texttt{./test\_dumux.sh}.
If everything works fine, a ParaView window with the result should open automatically.
