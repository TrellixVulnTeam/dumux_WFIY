%%%%%%%%%%%%%%%%%%%%%%%%%%%%%%%%%%%%%%%%%%%%%%%%%%%%%%%%%%%%%%%%%
% This file has been autogenerated from the LaTeX part of the   %
% doxygen documentation; DO NOT EDIT IT! Change the model's .hh %
% file instead!!                                                %
%%%%%%%%%%%%%%%%%%%%%%%%%%%%%%%%%%%%%%%%%%%%%%%%%%%%%%%%%%%%%%%%%


Adaption of the BOX scheme to the non-\/isothermal twophase flow model. This model implements a non-\/isothermal two-\/phase flow of two completely immiscible fluids $\alpha \in \{ w, n \}$. Using the standard multiphase Darcy approach, the mass conservation equations for both phases can be described as follows: \begin{eqnarray*} && \phi \frac{\partial (\varrho_\alpha S_\alpha )}{\partial t} - \text{div} \left\{ \varrho_\alpha \frac{k_{r\alpha}}{\mu_\alpha} \mbox{\bf K} (\text{grad}\, p_\alpha - \varrho_{\alpha} \mbox{\bf g}) \right\} - q_\alpha^\kappa = 0 \qquad \alpha \in \{w, n\} \end{eqnarray*} For the energy balance, local thermal equilibrium is assumed which results in one energy conservation equation for the porous solid matrix and the fluids: \begin{eqnarray*} && \phi \frac{\partial \left( \sum_\alpha \varrho_\alpha u_\alpha S_\alpha \right)}{\partial t} + \left( 1 - \phi \right) \frac{\partial (\varrho_s c_s T)}{\partial t} - \sum_\alpha \text{div} \left\{ \varrho_\alpha h_\alpha \frac{k_{r\alpha}}{\mu_\alpha} \mbox{\bf K} \left( \text{grad} \, p_\alpha - \varrho_\alpha \mbox{\bf g} \right) \right\} \\ &-& \text{div} \left( \lambda_{pm} \text{grad} \, T \right) - q^h = 0, \qquad \alpha \in \{w, n\}. \end{eqnarray*}

The equations are discretized using a fully-\/coupled vertex centered finite volume (box) scheme as spatial and the implicit Euler method as time discretization.

Currently the model supports choosing either $p_w$, $S_n$ and $T$ or $p_n$, $S_w$ and $T$ as primary variables. The formulation which ought to be used can be specified by setting the {\ttfamily Formulation} property to either {\ttfamily TwoPNIIndices::pWsN} or {\ttfamily TwoPIndices::pNsW}. By default, the model uses $p_w$, $S_n$ and $T$.

