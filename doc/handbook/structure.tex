\chapter{Structure}

We briefly describe the structure of \Dumux in terms 
of subdirectories, source files, and tests. For more details, 
the Doxygen documentation should be considered. 
\Dumux comes in form of a DUNE module \texttt{dune-mux}. 
It has a similar structure as other DUNE modules like \texttt{dune-grid}. 
The following subdirectories are within the module's root directory, 
from now on assumed to be \texttt{dune-mux}: 
\begin{itemize} 
\item \texttt{CMake}: the configuration options 
for building \Dumux using CMake. See the file \texttt{INSTALL.cmake} in 
the root directory \texttt{dune-mux} for details. Of course, 
it is also possible to use the DUNE buildsystem just like for the other 
DUNE modules.
\item \texttt{doc}: contains the Doxygen documentation in \texttt{doxygen}, 
this handbook in \texttt{handbook}, and the \Dumux logo in various formats in 
\texttt{logo}. The html documentation produced by Doxygen can be accessed as usual, 
namely, by opening \texttt{doc/doxygen/html/index.html} with a web browser. 
\item \texttt{dumux}: the \Dumux source files. See Section \ref{sec:dumux} for details. 
\item \texttt{test}: tests for each numerical model and the property system. 
See Section \ref{sec:test} for details. 
\item \texttt{tutorial}: contains the tutorials described in Chapter \ref{chp:tutorial}. 
\item \texttt{util}:   contains various auxiliary programs which are not
directly related to the core \Dumux libraries like programs to generate
tables for material laws, to fix the indentation of the source code,
etc.
\end{itemize}



\section{The directory \texttt{dumux}}\label{sec:dumux}

The directory \texttt{dumux} contains the \Dumux source files. It consists 
of the following subdirectories: 
\begin{itemize} 
\item \texttt{auxiliary}: 
general stuff like the property system and the time management for the 
fully coupled models. 

\item \texttt{boxmodels}:
the general fully implicit box method is contained in the subdirectory 
\texttt{boxscheme}, while each of the other subdirectories contains 
a derived specific numerical model. 

\item \texttt{diffusion}: numerical models to solve the pressure equation 
as part of the fractional flow formulation. The particular models are contained 
in corresponding subdirectories, which are distinguished by the respectively employed 
discretization scheme. 

\item \texttt{fractionalflow}:
the (non-compositional) fractional flow model, which utilizes the IMPES method 
contained in the subdirectory \texttt{impes}.

\item \texttt{functions}:
the Crouzeix--Raviart function implemented in the style of \texttt{dune-disc}'s P1 function. 

\item \texttt{fvgeometry}:
employed by the box method to extract the dual mesh geometry information out of the 
primal one. 

\item \texttt{io}: Additional in-/output possibilities like restart files 
and a VTKWriter extension. 

\item \texttt{material}: everything related to material parameters and 
constitutive equations. The base classes for matrix and fluid properties 
can be found in \texttt{property\_baseclasses.hh}, while the base classes 
for constitutive relations are in \texttt{twophaserelations.hh} and \texttt{multicomponentrelations.hh}.
Derived particular properties and relations are contained in 
\texttt{matrixproperties.hh} and in the subdirectories 
\texttt{fluids} and \texttt{constrel}. 


\item \texttt{nonlinear}: Newton's method.


\item \texttt{operators}: based on \texttt{dune-disc}, assembly operators for Crouzeix--Raviart 
elements and mimetic finite differences. 


\item \texttt{pardiso}: interface to the Pardiso direct solver library, \cite{Pardiso}. 


\item \texttt{shapefunctions}:  Crouzeix--Raviart element shape functions. 


\item \texttt{timedisc}: time discretization for the decoupled models. 


\item \texttt{transport}: numerical models to solve the pressure equation 
as part of the fractional flow formulation analogous to the \texttt{diffusion} 
directory. Moreover, the compositional decoupled models are included here. 


\end{itemize}



\section{The directory \texttt{test}}\label{sec:test}


The directory \texttt{test} contains a test for each numerical model and for 
the property system. The subfolders \texttt{1p},  \texttt{1p2c},  \texttt{2p},  \texttt{2p2c},  
\texttt{2p2cni},  \texttt{2pni},  and \texttt{richards} contain tests for the fully 
coupled models, while \texttt{diffusion}, \texttt{fractionalflow}, and \texttt{transport} 
correspond to the decoupled models. Each subdirectory 
contains one or more program files \texttt{test\_*.cc}, where \texttt{*} usually is the 
name of the folder. Moreover, the problem definitions can be found 
in the \texttt{*problem.hh} files. Simply executing the tests should either run the 
full test or give a list of required command line arguments. After test execution, 
VTK output files should have been generated. 
For more detailed descriptions of the tests, the problem definitions and their corresponding 
Doxygen documentation should be considered. 

